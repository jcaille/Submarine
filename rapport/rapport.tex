%%%%%%%%%%%%%%%%%%%%%%%%%%%%%%%%%%%%%%%%%
% Journal Article
% LaTeX Template
% Version 1.3 (9/9/13)
%
% This template has been downloaded from:
% http://www.LaTeXTemplates.com
%
% Original author:
% Frits Wenneker (http://www.howtotex.com)
%
% License:
% CC BY-NC-SA 3.0 (http://creativecommons.org/licenses/by-nc-sa/3.0/)
%
%%%%%%%%%%%%%%%%%%%%%%%%%%%%%%%%%%%%%%%%%

%----------------------------------------------------------------------------------------
%	PACKAGES AND OTHER DOCUMENT CONFIGURATIONS
%----------------------------------------------------------------------------------------

\documentclass[twoside]{article}

\usepackage{lipsum} % Package to generate dummy text throughout this template

\usepackage[T1]{fontenc} % Use 8-bit encoding that has 256 glyphs
\usepackage[utf8]{inputenc}

\usepackage[hmarginratio=1:1,top=32mm,columnsep=20pt]{geometry} % Document margins
\usepackage{multicol} % Used for the two-column layout of the document
\usepackage[hang, small,labelfont=bf,up,textfont=it,up]{caption} % Custom captions under/above floats in tables or figures
\usepackage{booktabs} % Horizontal rules in tables
\usepackage{float} % Required for tables and figures in the multi-column environment - they need to be placed in specific locations with the [H] (e.g. \begin{table}[H])
\usepackage{hyperref} % For hyperlinks in the PDF

\usepackage{lettrine} % The lettrine is the first enlarged letter at the beginning of the text
\usepackage{paralist} % Used for the compactitem environment which makes bullet points with less space between them

\usepackage{abstract} % Allows abstract customization
\renewcommand{\abstractnamefont}{\normalfont\bfseries} % Set the "Abstract" text to bold
\renewcommand{\abstracttextfont}{\normalfont\small\itshape} % Set the abstract itself to small italic text

\usepackage{titlesec} % Allows customization of titles
\renewcommand\thesection{\Roman{section}} % Roman numerals for the sections
\renewcommand\thesubsection{\Roman{subsection}} % Roman numerals for subsections
\titleformat{\section}[block]{\large\scshape\centering}{\thesection.}{1em}{} % Change the look of the section titles
\titleformat{\subsection}[block]{\large}{\thesubsection.}{1em}{} % Change the look of the section titles

\usepackage{fancyhdr} % Headers and footers
\pagestyle{fancy} % All pages have headers and footers
\fancyhead{} % Blank out the default header
\fancyfoot{} % Blank out the default footer
\fancyhead[C]{Améliorations d'images sous-marines. Rapport de Projet} % Custom header text
\fancyfoot[RO,LE]{\thepage} % Custom footer text

%----------------------------------------------------------------------------------------
%	TITLE SECTION
%----------------------------------------------------------------------------------------

\title{\vspace{-15mm}\fontsize{24pt}{10pt}\selectfont\textbf{Amélioration d'images sous-marines}} % Article title

\author{
\large
\textsc{Jean Caillé, Florian Denis}\\[2mm] % Your name
\normalsize Télécom ParisTech - SI241 \\ % Your institution
\normalsize \href{mailto:jean.caille@polytechnique.edu}{jean.caille@polytechnique.edu} - \href{mailto:florian.denis@polytechnique.edu}{florian.denis@polytechnique.edu}  % Your email address
\vspace{-5mm}
}
\date{}

%----------------------------------------------------------------------------------------

\begin{document}

\maketitle % Insert title

\thispagestyle{fancy} % All pages have headers and footers

%----------------------------------------------------------------------------------------
%	ABSTRACT
%----------------------------------------------------------------------------------------

\begin{abstract}

Les images sous marines sont par définitions acquises dans un environnement ou les conditions de visibilités sont particulièrement mauvaises. Ainsi, les images obtenues par des appareils classiques sont fortement dégradées. L'hétérogénéité du milieu, la diffusion de la lumière par les particules en suspension et la non-transparence de l'eau contribuent chacun à une prise de vue s'éloignant de la réalité. L'article que nous avons étudié montre qu'il est possible de supprimer ces dégradations dans certians cas, en utilisant des hypothèses minimes quand à la scène observée. Cette méthode est de plus applicable à un grand nombre d'exemple car elle ne nécessite qu'une image en entrée.\\
Nous avons étudié, implémenté et mesuré les performances de la méthodes proposées par les auteurs de l'article. Dan ce rapport, nous nous attacherons à décrire d'une part la technique employée, et nous décrirons les outils utilisés pour l'implémentation. Nous analyserons ensuite nos résultats et les comparerons avec ceux obtenus par les auteurs. Finalement, nous tenterons de suggérer des améliorations à cette méthode.\\

\end{abstract}

%----------------------------------------------------------------------------------------
%	ARTICLE CONTENTS
%----------------------------------------------------------------------------------------

\begin{multicols}{2} % Two-column layout throughout the main article text

\section{Introduction et Motivation}
Aujoud'hui, le grand public commence à s'équiper d'appareils photos résistants à l'eau (GoPro, ...). Les images acquises dans ces conditions difficiles sont souvents dégradées. Toutefois, ce même matériel est suffisement sensible pour permettre la restauration des images, permettant alors d'améliorer grandement la qualité des images rendues.\\
Nous avons choisi d'effectuer ce projet car les résultats promis par l'article et la description qui en était faite nous semblait intéressants. Plusieurs concepts variés, très utilisés en images sont introduits (Balance des blancs et Gray-World, Fusion par pyramide, ...). De plus, les techniques décritent dans le papier sont applicable à de nombreux problèmes de restauration d'images, en particulier les images dégradées par un brouillard ou par la polution aérienne. Finalement, ce projet nous permettait de tester notre implémentation dans des cas réelles (typiquement : des images sous-marines trouvées sur internet). En effet, contrairement à certains projets ou les données d'entrées sont nombreuses, ou dans un format particulier, la technique décrite dans l'article peut s'appliquer aux photos individuelles.\\
L’article propose un post-traitement basé sur la fusion d’images afin d’améliorer la qualité de la prise de vue. Les applications d’un tel traitement sont nombreuses, tant pour l’affichage et la visualisation de ces images que pour des problématiques de visions plus intéressantes (tels l’extraction de points clés, la détection de contours et la reconnaissance de forme).\\

%------------------------------------------------

\section{Méthode}

La méthode décrite par l’article consiste en la fusion de deux images générées (appelées images d'entrées) à partir de l’image initiale. Les auteurs supposent que les dégradations de la prise de vue sous-marine sont du d’une part la balance des couleurs ainsi que les contrastes de la scène. Ainsi les deux images qui seront fusionnées sont chacune créées afin de résoudre un de ces problèmes.\\

%---------------------------------------------------> SUB SECTION Balance des blancs
\subsection{Balance des blancs}
La première image du processus de fusion est créée afin de corriger les problèmes de balance de blancs qui existent lors des prises de vue sous marines. Comme nous pouvons le voir sur la figure, les informations de couleurs semblent à première vue être perdues. [[FIGURE]]\\
Nous pouvons toutefois en post-traietement corriger la balance des blancs. L'article suggère plusieurs méthodes, mais aucune n'est décrite en particulier. Nous avons donc choisi d'implémenter la méthode dite de \emph{Gray-World}. L'idée sous-jacente de cette technique est de trouver l'\emph{illuminant} $i$, c'est à dire la couleur de la lumière éclairant la scène. Une fois $i$ trouvé, nous pouvons "soustraire" cette couleur à l'ensemble des pixels de la scène pour retrouver les couleurs réelles (\emph{i.e.} en lumière blanche).\\
L'hypothèse \emph{Gray-World} suppose que la couleur moyenne d'une scène est un niveaux de gris $g$. Ainsi si on mesure $m$ la moyenne réelle des pixels de l'image : $$m = \frac{1}{N}\sum_{x \in image}I(x)$$ On peut calculer $g$ comme $$g = \frac{m_r + m_g + m_b}{3}$$ L'article suggère une correction de la luminosité de scène pour compenser la perte de luminosité due au milieu $$g' = 0.5 + \lambda g$$ où $\lambda$ est un paramètre dont la valeur recommandée par les auteurs est $0.2$. On en déduit alors la couleur de l'illuminant $$ i = \frac{m}{g}$$ Finalement, on corrige la couleur de la scène en modifiant la valeur du pixel $$I(x) = \frac{I(x)}{i}$$\\
Nous avons ainsi corrigé l'image de telle sorte que la somme des couleurs soit effectivement une nuance de gris. L'image obtenu sera la première entrée du processus de fusion.

%---------------------------------------------------> SUB SECTION Amélioration du Contraste
\subsection{Amélioration du Contraste}
La seconde entrée du procussus de fusion est quand à elle créée pour corriger les problèmes de contrastes dus à la diffusion de la lumière dans le milieu aquatique. Elle est obtenue à paritr de l'image où les couleurs ont été corrigées. Elle est obtenue en deux étapes : réduction du bruit puis amélioration du constraste.\\
La réduction du bruit est effectué à l'aide d'un simple filtre bilatéral. Ce filtre non linéaire remplace l'intensité de chaque pixel par une moyenne pondéré des pixels avoisinants, où les poids dépendent à la fois de la distance au pixel central, mais également de la différence d'intensité entre les pixels considérés. Ce bruit a la propriété de conserver les contours tout en supprimant correctement le bruit. Il est important de noter que si l'article conseille un débruitage cohérent dans le temps (pouvant donc s'appliquer à la vidéo), cela n'est pas nécessaire pour les applications à des images simples. [[FIGURE]]\\
Il existe de nombreuses méthodes permettant de réhausser le contraste de l'image, en particulier la méthode de \emph{Local Adaptative histogram Equalization} qui égalise localement des histogrammes, c'est-à-dire une transformation locale du constraste basée sur le voisinage du pixel. Contrairement à une méthode globale, on ne court pas le risque de saturer l'image tout en gardant une plage dynamique maximum. Toutefois, il est important de noter que cette technique est très sensible au bruit. Si le filtre bilatéral appliqué n'est pas assez puissant, il peut être nécessaire d'employer une technique plus robuste pour augmenter le constraste. Une solution consiste à employer une méthode similaire nommée \emph{Constrast Limited Adaptative Histogram Equalization (CLAHE)}. Dans cette méthode, le contraste est limité, et les pics de l'histogrammes sont redistribués uniformément dans l'histogramme.

%---------------------------------------------------> SUB SECTION Poids de fusions
\subsection{Poids de fusions}
Une fois les deux entrées générées, il est nécessaire de les fusionner de manière intelligente. Une simple moyenne des images (fusion naïve) n'est pas adaptée, en effet, sur certaines zones (contours, zones visuellement proéminentes, ...) il est nécessaire de faire ressortir le contraste (et donc l'entrée numéro 2) au détriment des couleurs, alors que dans les zones constantes (aplats de couleur), il vaut mieux mettre en avant la correction de couleur. La fusion va donc s'effectuer en fonction d'un ou plusieurs poids, qui dénotterons l'importance que l'on souhaite appliquer à chaque image d'entrée dans l'image résultante, et ce pour tous les pixels considérés. 

%------------------------------------------------------------> SUB SUB SECTION Poid 1
\subsubsection{Poids Laplacien}
Il se calcul en appliquand un filtre laplacien sur chaque entrées du filtre. Il est censé mettre en valeur les contours et zones texturées, mais est peu adapté aux problèmes de restaurations d'images sous-marines, car il ne fait pas la différence entre zones de lumnisoté constante et zones de luminosité légèrement croissante (ou décroissante)

%------------------------------------------------------------> SUB SUB SECTION Poids 2
\subsubsection{Poids de Contraste Local}
Ce poids a été créé pour mettre en avant les zones contrastées. Il se calcul en chaque pixel par la différence de la valeur de l'image I avec la valeur de I filtré par un passe-bas : $$ W_{LC}(x,y) = \|I(x,y) - I_{f}(x,y)\| $$
L'article recommande l'utilisation pour le filtre passe-bas d'un noyeau binomial séparable de taille $5$ x $5$. Ce filtre est en effet une bonne approximation d'une Gaussienne, tout en restant algorithmiquement peu couteux à calculer.

%------------------------------------------------------------> SUB SUB SECTION Poids de Saillance
\subsubsection{Poids de Saillance}
Le poids de Saillanche cherche à mettre en valeur les objets visuellement proéminants qui peuvent exister dans une scène sous-marines. Une technique pour trouver la saillance d'un point et de comparer sa valeure avec une moyenne des valeurs des pixels l'entourant. Ce comportement est similaire au concept biologique de \emph{center-surround constrast}, ou contraste centre-contour. Toutefois, l'article précise que ce poids favorise les zones lumineuses, et pour protéger les tons moyens, les auteurs suggèrent l'utilisation du poids d'exposition.

%------------------------------------------------------------> SUB SUB SECTION Poids d'exposition
\subsubsection{Poids d'exposition}
Cette carte de poids cherche à estimer si le pixel considéré est bien exposé, i.e. si sa luminosité est proche de la valeur moyenne $0.5$. On utilise ensuite une gaussienne pour obtenir la valeur du poids : $$W_{exp}(x,y) = \exp{-\frac{(I(x,y) - 0.5)^2}{2\sigma^2}}$$

%------------------------------------------------------------> SUB SUB SECTION Normalisation des poids
 \subsubsection{Normalisation des poids}
Plusieurs poids sont suggérés, mais seulement deux (un par entrée) sont utilisées dans le processus de fusion. Pour normalizer la contribution de chaque image, on modifie les poids de tel sorte que la somme des deux cartes de poids soit constante et égales à $1$. Cette modification est purement locale, et ne dépend que de la valeur des poids au pixel considéré.

%---------------------------------------------------> SUB SECTION Fusion par pyramide Laplacienne
\subsection{Fusion par pyramide Laplacienne}
	
\section{Implémentation}

\section{Résultats}

\section{Ouvertures et Amélioratiosn possibles}

%----------------------------------------------------------------------------------------
%	REFERENCE LIST
%----------------------------------------------------------------------------------------

\begin{thebibliography}{99} % Bibliography - this is intentionally simple in this template

\bibitem[Figueredo and Wolf, 2009]{Figueredo:2009dg}
Figueredo, A.~J. and Wolf, P. S.~A. (2009).
\newblock Assortative pairing and life history strategy - a cross-cultural
  study.
\newblock {\em Human Nature}, 20:317--330.
 
\end{thebibliography}

%----------------------------------------------------------------------------------------

\end{multicols}

\end{document}
